% This file is part of TeX by Topic
% Copyright 2007 Victor
% see file TeXbyTopic.tex for copying conditions

This chapter gives the list of all primitives of
\TeX. After each control sequence
the grammatical category of the command or parameter
is given, plus a short description. For some  commands
the syntax of their use is given.

For parameters the class to which they belong is given.
Commands that have no grammatical category
in \TeXbook\ are denoted either
`\gr{expandable command}' or
`\gr{primitive command}' in this list.

Grammatical terms such as \gr{equals} and \gr{optional space}
are explained in Chapter~\ref{gramm}.

\begin{glossinventory}

\item [\cs{-}]
      \gr{horizontal command} 
      Discretionary hyphen; this is
      equivalent to \cs{discretionary}""\verb|{-}{}{}|.
      Can be used to indicate hyphenatable points in a word. 
Chapter~\ref{cschap:-}.

\item [\cs{char32}]
      \gr{horizontal command} 
      Control space.
      Insert the same amount of space as a space token would
 \alt
      if \cs{spacefactor}${}=1000$.
Chapter~\ref{cschap:char32},\ref{cschap:char322}.

\item [\cs{char47}]
      \gr{primitive command} 
      Italic correction: insert a kern specified by the
      preceding character.
      Each character has an italic correction, possibly zero,
      specified in the \n{tfm} file.
      For slanted fonts this compensates for overhang.
Chapter~\ref{cschap:char47}.

\item [\cs{above\gr{dimen}}]
      \gr{generalized fraction command} 
      Fraction with specified bar width. 
Chapter~\ref{cschap:above}.

\item [\cs{abovedisplayshortskip}]
      \gr{glue parameter}
      Glue above a display if the line preceding the display was short.
Chapter~\ref{cschap:abovedisplayshortskip}.

\item [\cs{abovedisplayskip}]
      \gr{glue parameter}
      Glue above a display.
Chapter~\ref{cschap:abovedisplayskip}.

\item [\cs{abovewithdelims\gr{delim$_1$}\gr{delim$_2$}\gr{dimen}}]
      \gr{generalized fraction command}
      Generalized fraction with delimiters.
Chapter~\ref{cschap:abovewithdelims}.

\item [\cs{accent\gr{8-bit number}\gr{optional assignments}\gr{character}}]
      \gr{horizontal command}
      Command to place accents on characters.\alt
Chapter~\ref{cschap:accent}.

\item [\cs{adjdemerits}]
      \gr{integer parameter}
      Penalty for adjacent not visually compatible lines. 
      Default~\n{10$\,$000} in plain \TeX.
Chapter~\ref{cschap:adjdemerits}.

\item [\cs{advance\gr{numeric variable}\gr{optional \n{by}}\gr{number}}]
      \gr{arithmetic assignment}
      Arithmetic command to increase or decrease a 
      \gr{numeric variable}, that is,
 \alt
      a \gr{count variable}, \gr{dimen variable}, \gr{glue variable}, 
      or \gr{muglue variable}.
Chapter~\ref{cschap:advance},\ref{cschap:advance2}.

\item [\cs{afterassignment\gr{token}}]
      \gr{primitive command}
      Save the next token for execution after the next assignment.
      Only one token can be saved this way.
Chapter~\ref{cschap:afterassignment}.

\item [\cs{aftergroup\gr{token}}]
      \gr{primitive command}
      Save the next token for insertion after the current group.
      Several tokens can be saved this way.
Chapter~\ref{cschap:aftergroup}.

\item [\cs{atop\gr{dimen}}]
      \gr{generalized fraction command}
      Place objects over one another.
Chapter~\ref{cschap:atop}.

\item [\cs{atopwithdelims\gr{delim$_1$}\gr{delim$_2$}}]
      \gr{generalized fraction command}
      Place objects over one another with delimiters.
Chapter~\ref{cschap:atopwithdelims}.

\item [\cs{badness}]
      \gr{internal integer} 
      (\TeX3 only) 
      Badness of the most recently constructed box.
Chapter~\ref{cschap:badness}.

\item [\cs{baselineskip}]
      \gr{glue parameter} 
      The `ideal' baseline distance between neighbouring 
      boxes on a vertical list;  \n{12pt} in plain \TeX.
Chapter~\ref{cschap:baselineskip}.

\item [\cs{batchmode}]
      \gr{interaction mode assignment}
      \TeX\ patches errors itself 
 \alt
      and performs an emergency stop on serious errors 
      such as missing input files,
      but no terminal output is generated.
Chapter~\ref{cschap:batchmode}.

\item [\cs{begingroup}]
      \gr{primitive command}
      Open a group that must be closed with \verb-\endgroup-.
Chapter~\ref{cschap:begingroup}.

\item [\cs{belowdisplayshortskip}]
      \gr{glue parameter}
      Glue below a display if the line preceding the display was short.
Chapter~\ref{cschap:belowdisplayshortskip}.

\item [\cs{belowdisplayskip}]
      \gr{glue parameter}
      Glue below a display.
Chapter~\ref{cschap:belowdisplayskip}.

\item [\cs{binoppenalty}]
      \gr{integer parameter}
      Penalty for breaking after a binary operator not enclosed in
      a subformula.
      Plain \TeX\ default:~\n{700}.
Chapter~\ref{cschap:binoppenalty}.

\item [\cs{botmark}]
      \gr{expandable command}
      The last mark on the current page.
Chapter~\ref{cschap:botmark}.

\item [\cs{box\gr{8-bit number}}]
      \gr{box}
      Use a box register, emptying it. 
Chapter~\ref{cschap:box}.

\item [\cs{boxmaxdepth}]
      \gr{dimen parameter}
      Maximum allowed depth of boxes.
      Default~\cs{maxdimen} in plain \TeX.
Chapter~\ref{cschap:boxmaxdepth}.

\item [\cs{brokenpenalty}]
      \gr{integer parameter}
      Additional penalty for breaking a page after a hyphenated line. 
      Default~\n{100} in plain \TeX.
Chapter~\ref{cschap:brokenpenalty}.

\item [\cs{catcode\gr{8-bit number}}]
      \gr{internal integer}; the control sequence itself
      is a~\gr{codename}.
      Access category codes.
Chapter~\ref{cschap:catcode}.

\item [\cs{char\gr{number}}]
      \gr{character}
      Explicit denotation of a character to be typeset. 
Chapter~\ref{cschap:char}.

\item [\cs{chardef\gr{control sequence}\gr{equals}\gr{number}}]
      \gr{shorthand definition}
      Define a control sequence to be a synonym for
      a~character code.
Chapter~\ref{cschap:chardef}.

\item [\cs{cleaders}]
      \gr{leaders}
      As \verb=\leaders=, but with box leaders 
      any excess space is split into equal glue items
 \alt
      before and after the leaders.
Chapter~\ref{cschap:cleaders}.

\item [\cs{closein\gr{4-bit number}}]
      \gr{primitive command}
      Close an input stream.
Chapter~\ref{cschap:closein}.

\item [\cs{closeout\gr{4-bit number}}]
      \gr{primitive command}
      Close an output stream.
Chapter~\ref{cschap:closeout}.

\item [\cs{clubpenalty}]
      \gr{integer parameter}
      Additional penalty for breaking a page after the first line of a paragraph. 
      Default~\n{150} in plain \TeX.
Chapter~\ref{cschap:clubpenalty}.

\item [\cs{copy\gr{8-bit number}}]
      \gr{box}
      Use a box register and retain the contents. 
Chapter~\ref{cschap:copy}.

\item [\cs{count\gr{8-bit number}}]
      \gr{internal integer}; the control sequence itself
      is a~\gr{register prefix}.
      Access count registers. 
Chapter~\ref{cschap:count}.

\item [\cs{countdef\gr{control sequence}\gr{equals}\gr{8-bit number}}]
      \gr{shorthand definition}; the control sequence
      itself is a~\gr{registerdef}.
      Define a control sequence to be a synonym for
      a~\cs{count} register.
Chapter~\ref{cschap:countdef}.

\item [\cs{cr}]
      \gr{primitive command}
      Terminate an alignment line.
Chapter~\ref{cschap:cr}.

\item [\cs{crcr}]
      \gr{primitive command}
      Terminate an alignment line if it has 
      not already been terminated by~\cs{cr}.
Chapter~\ref{cschap:crcr}.

\item [\cs{csname}]
      \gr{expandable command}
      Start forming the name of a control sequence.
      Has to be balanced with \cs{endcsname}.
Chapter~\ref{cschap:csname}.

\item [\cs{day}]
      \gr{integer parameter}
      The day of the current job.
Chapter~\ref{cschap:day}.

\item [\cs{deadcycles}]
      \gr{special integer}
      Counter that keeps track of how many times 
      the output routine has been called without a \cs{shipout} 
      taking place. 
      If this number reaches \cs{maxdeadcycles} \TeX\
      gives an error message. Plain \TeX\ default:~\n{25}.
Chapter~\ref{cschap:deadcycles}.

\item [\cs{def}]
      \gr{def} Start a macro definition.
Chapter~\ref{cschap:def}.

\item [\cs{defaulthyphenchar}]
      \gr{integer parameter}
      Value of \cs{hyphenchar} when a font is loaded.
      Default value in plain \TeX\ is~\verb>`\->.
Chapter~\ref{cschap:defaulthyphenchar},\ref{cschap:defaulthyphenchar2}.

\item [\cs{defaultskewchar}]
      \gr{integer parameter}
      Value of \cs{skewchar} when a font is loaded.
      Default value in plain \TeX\ is~\n{-1}.
Chapter~\ref{cschap:defaultskewchar}.

\item [\cs{delcode\gr{8-bit number}}]
      \gr{internal integer}; the control sequence itself
      is a~\gr{codename}.
      Access the
      code specifying how a character should be used as delimiter 
      after \cs{left} or \cs{right}.
Chapter~\ref{cschap:delcode}.

\item [\cs{delimiter\gr{27-bit number}}]
      \gr{math character}
      Explicit denotation of a delimiter.
Chapter~\ref{cschap:delimiter}.

\item [\cs{delimiterfactor}]
      \gr{integer parameter}
      1000 times the part of a delimited formula that should be
      covered by a delimiter.
      Plain \TeX\ default:~\n{901}.
Chapter~\ref{cschap:delimiterfactor}.

\item [\cs{delimitershortfall}]
      \gr{integer parameter}
      Size of the part of a delimited formula that is allowed 
      to go uncovered by a delimiter.
      Plain \TeX\ default:~\n{5pt}.
Chapter~\ref{cschap:delimitershortfall}.

\item [\cs{dimen\gr{8-bit number}}]
      \gr{internal dimen}; the control sequence itself
      is a~\gr{register prefix}.
      Access dimen registers.
Chapter~\ref{cschap:dimen}.

\item [\cs{dimendef\gr{control sequence}\gr{equals}\gr{8-bit number}}]
      \gr{shorthand definition}; the control sequence
      itself is a~\gr{registerdef}.
      Define a control sequence to be a synonym for
      a~\cs{dimen} register.
Chapter~\ref{cschap:dimendef}.

\item [\cs{discretionary\SerifFont\italic\lb pre-break\rb\lb post-break\rb\lb no-break\rb{}}]
      \gr{horizontal command}
      Specify the way a character sequence is split up at a line break.
Chapter~\ref{cschap:discretionary}.

\item [\cs{displayindent}]
      \gr{dimen parameter}
      Distance by which the box, in which the display 
      is centred, is indented owing to hanging indentation.
      This value is set automatically for each display.
Chapter~\ref{cschap:displayindent}.

\item [\cs{displaylimits}]
      \gr{primitive command}
      Restore default placement for limits.
Chapter~\ref{cschap:displaylimits}.

\item [\cs{displaystyle}]
      \gr{primitive command}
      Select the display style of math typesetting.
Chapter~\ref{cschap:displaystyle}.

\item [\cs{displaywidowpenalty}]
      \gr{integer parameter}
      Additional penalty for breaking a page before the last line 
      above a display formula. 
      Default~\n{50} in plain \TeX.
Chapter~\ref{cschap:displaywidowpenalty}.

\item [\cs{displaywidth}]
      \gr{dimen parameter}
      Width of the box in which the display is centred.
      This value is set automatically for each display.
Chapter~\ref{cschap:displaywidth}.

\item [\cs{divide\gr{numeric variable}\gr{optional \n{by}}\gr{number}}]
      \gr{arithmetic assignment}
      Arithmetic command to divide a \gr{numeric variable}
      (see \cs{advance}).
Chapter~\ref{cschap:divide}.

\item [\cs{doublehyphendemerits}]
      \gr{integer parameter}
      Penalty for consecutive lines ending with a hyphen. 
      Default~\n{10$\,$000} in plain \TeX.
Chapter~\ref{cschap:doublehyphendemerits}.

\item [\cs{dp\gr{8-bit number}}]
      \gr{internal dimen}; the control sequence itself
      is a~\gr{box dimension}.
      Depth of the box in a box register. 
Chapter~\ref{cschap:dp}.

\item [\cs{dump}]
      \gr{vertical command}
      Dump a format file; possible only in \IniTeX, 
      not allowed inside a group.

Chapter~\ref{cschap:dump}.

\item [\cs{edef}]
      \gr{def}
      Start a macro definition; 
      the replacement text is expanded at definition time.
Chapter~\ref{cschap:edef}.

\item [\cs{else}]
      \gr{expandable command}
      Select
      \gr{false text} of a conditional 
      or default case of \cs{ifcase}.
Chapter~\ref{cschap:else}.

\item [\cs{emergencystretch}]
      \gr{dimen parameter} 
      (\TeX3 only) 
      Assumed extra stretchability in lines of a paragraph
      in third pass of the line-breaking algorithm.
Chapter~\ref{cschap:emergencystretch}.

\item [\cs{end}]
      \gr{vertical command}
      End this run.
Chapter~\ref{cschap:end}.

\item [\cs{endcsname}]
      \gr{expandable command}
      Delimit the name of a control sequence that was begun
      with \cs{csname}.
Chapter~\ref{cschap:endcsname}.

\item [\cs{endgroup}]
      \gr{primitive command}
      End a group that was opened with \verb-\begingroup-.
Chapter~\ref{cschap:endgroup}.

\item [\cs{endinput}]
      \gr{expandable command}
      Terminate inputting the current file after the current line.
Chapter~\ref{cschap:endinput}.

\item [\cs{endlinechar}] 
      \gr{integer parameter}
      The character code of the end-of-line character 
      appended to input lines.
      \IniTeX\ default:~\n{13}.
Chapter~\ref{cschap:endlinechar}.

\item [\cs{eqno\gr{math mode material}\n{\char36\char36}}]
      \gr{eqno}
      Place a right equation number in a display formula.
Chapter~\ref{cschap:eqno}.

\item [\cs{errhelp}]
      \gr{token parameter}
      Tokens that will be displayed if the user 
      asks for help after an \cs{err\-message}.
Chapter~\ref{cschap:errhelp}.

\item [\cs{errmessage\gr{general text}}]
      \gr{primitive command}
      Report an error and give the user opportunity to act.
Chapter~\ref{cschap:errmessage}.

\item [\cs{errorcontextlines}]
      \gr{integer parameter} 
      (\TeX3 only)
      Number of additional context lines shown in error messages.
Chapter~\ref{cschap:errorcontextlines}.

\item [\cs{errorstopmode}]
      \gr{interaction mode assignment}  
      Ask for user input on the occurrence of an error.
Chapter~\ref{cschap:errorstopmode}.

\item [\cs{escapechar}] 
      \gr{integer parameter}
      Number of the character that is  used 
      when control sequences are being converted
      into character tokens.
      \IniTeX\ default:~\n{92}.
Chapter~\ref{cschap:escapechar}.

\item [\cs{everycr}]
      \gr{token parameter}
      Token list inserted after every \cs{cr} or non-redundant \cs{crcr}.
Chapter~\ref{cschap:everycr}.

\item [\cs{everydisplay}]
      \gr{token parameter}
      Token list inserted at the start of a display.
Chapter~\ref{cschap:everydisplay}.

\item [\cs{everyhbox}]
      \gr{token parameter}
      Token list inserted at the start of a horizontal box.
Chapter~\ref{cschap:everyhbox}.

\item [\cs{everyjob}]
      \gr{token parameter}
      Token list inserted at the start of each job.
Chapter~\ref{cschap:everyjob}.

\item [\cs{everymath}]
      \gr{token parameter}
      Token list inserted at the start of non-display math.
Chapter~\ref{cschap:everymath}.

\item [\cs{everypar}]
      \gr{token parameter}
      Token list inserted in front of paragraph text.
Chapter~\ref{cschap:everypar}.

\item [\cs{everyvbox}]
      \gr{token parameter}
      Token list inserted at the start of a vertical box.
Chapter~\ref{cschap:everyvbox}.

\item [\cs{exhyphenpenalty}]
      \gr{integer parameter}
      Penalty for breaking a horizontal line at a discretionary
      in the special case where the prebreak text is empty. 
      Default~\n{50} in plain \TeX.
Chapter~\ref{cschap:exhyphenpenalty}.

\item [\cs{expandafter}] 
      \gr{expandable command}
      Take the next two tokens and place the 
      expansion  of the second after the first.
Chapter~\ref{cschap:expandafter}.

\item [\cs{fam}]
      \gr{integer parameter}
      The number of the current font family.
Chapter~\ref{cschap:fam}.

\item [\cs{fi}]
      \gr{expandable command}
      Closing delimiter for all conditionals.
Chapter~\ref{cschap:fi}.

\item [\cs{finalhyphendemerits}]
      \gr{integer parameter}
      Penalty added when the penultimate line of a 
      paragraph ends with a hyphen. 
      Plain \TeX\ default~\n{5000}.
Chapter~\ref{cschap:finalhyphendemerits}.

\item [\cs{firstmark}]
      \gr{expandable command}
      The first mark on the current page.
Chapter~\ref{cschap:firstmark}.

\item [\cs{floatingpenalty}]
      \gr{integer parameter}
      Penalty amount added to \cs{insertpenalties}
 \alt
      when an insertion is split.
Chapter~\ref{cschap:floatingpenalty}.

\item [\cs{font\gr{control sequence}\gr{equals}\gr{file name}\gr{at clause}}]
      \gr{simple assignment}
      Associate a control sequence with a \n{tfm} file.
      When used on its own, this control sequence is a \gr{font},
      denoting the current font.
Chapter~\ref{cschap:font}.

\item [\cs{fontdimen\gr{number}\gr{font}}]
      \gr{internal dimen}
      Access various parameters of fonts.
Chapter~\ref{cschap:fontdimen}.

\item [\cs{fontname\gr{font}}]
      \gr{primitive command}
      The external name of a font.
Chapter~\ref{cschap:fontname}.

\item [\cs{futurelet\gr{control sequence}\gr{token$_1$}\gr{token$_2$}}]
      \gr{let assignment}
      Assign the meaning of \gr{token$_2$} to the
      \gr{control sequence}.
Chapter~\ref{cschap:futurelet}.

\item [\cs{gdef}]
      \gr{def}
      Synonym for \verb-\global\def-.
Chapter~\ref{cschap:gdef}.

\item [\cs{global}]
      \gr{prefix}
      Make the next definition, arithmetic statement,
      or assignment global.
Chapter~\ref{cschap:global},\ref{cschap:global2}.

\item [\cs{globaldefs}]
      \gr{integer parameter}
      Override \cs{global} specifications: a positive value of this
      parameter makes all assignments global, a negative value
      makes them local.
Chapter~\ref{cschap:globaldefs}.

\item [\cs{halign\gr{box specification}\lb\gr{alignment material}\rb{}}]
      \gr{vertical command}
      Horizontal alignment.
      Display alignment:
      \begin{disp}\n{\$\$}\cs{halign}\gr{box specification}\lb\n{...}\rb
      \gr{optional assignments}\n{\$\$}\end{disp}
Chapter~\ref{cschap:halign}.

\item [\cs{hangafter}]
      \gr{integer parameter}
      If positive, this denotes the number of lines 
      before indenting starts; 
      if negative, its absolute value is the number 
      of indented lines starting with the first line of the paragraph. 
      The default value of~1 is restored after every paragraph.
Chapter~\ref{cschap:hangafter}.

\item [\cs{hangindent}]
      \gr{dimen parameter}
      If positive, this indicates indentation from the left margin; 
      if negative, this is the negative of the indentation 
      from the right margin. 
      The default value of~\n{0pt} is restored after every paragraph.
Chapter~\ref{cschap:hangindent}.

\item [\cs{hbadness}]
      \gr{integer parameter}
      Threshold below which \TeX\ does not report an underfull 
      or overfull  horizontal box.
      Plain \TeX\ default:~\n{1000}.
Chapter~\ref{cschap:hbadness}.

\item [\cs{hbox\gr{box specification}\lb\gr{horizontal material}\rb}]
      \gr{box}
      Construct a horizontal box. 
Chapter~\ref{cschap:hbox}.

\item [\cs{hfil}]
      \gr{horizontal command}
      Horizontal skip equivalent to \verb-\hskip 0cm plus 1fil-.
Chapter~\ref{cschap:hfil}.

\item [\cs{hfill}]
      \gr{horizontal command}
      Horizontal skip equivalent to \verb-\hskip 0cm plus 1fill-.
Chapter~\ref{cschap:hfill}.

\item [\cs{hfilneg}]
      \gr{horizontal command}
      Horizontal skip equivalent to \verb-\hskip 0cm minus 1fil-.
Chapter~\ref{cschap:hfilneg}.

\item [\cs{hfuzz}]
      \gr{dimen parameter}
      Excess size that \TeX\ tolerates before it considers  
      a horizontal box overfull.
      Plain \TeX\ default:~\n{0.1pt}.
Chapter~\ref{cschap:hfuzz}.

\item [\cs{hoffset}]
      \gr{dimen parameter}
      Distance by which the page is shifted to the right 
      of the reference point which is at one inch from
      the left margin.
Chapter~\ref{cschap:hoffset}.

\item [\cs{holdinginserts}]
      \gr{integer parameter} 
      (only \TeX3) 
      If this is positive, insertions are not placed in their boxes 
      when the \cs{output} tokens are inserted.
Chapter~\ref{cschap:holdinginserts}.

\item [\cs{hrule}]
      \gr{vertical command}
      Rule that spreads in horizontal direction.
Chapter~\ref{cschap:hrule}.

\item [\cs{hsize}]
      \gr{dimen parameter}
      Line width used for text typesetting inside a vertical box.
Chapter~\ref{cschap:hsize},\ref{cschap:hsize2}.

\item [\cs{hskip\gr{glue}}] 
      \gr{horizontal command}
      Insert in horizontal mode a glue item.
Chapter~\ref{cschap:hskip}.

\item [\cs{hss}]
      \gr{horizontal command}
      Horizontal skip equivalent to \verb-\hskip 0cm plus 1fil minus 1fil-.
Chapter~\ref{cschap:hss}.

\item [\cs{ht\gr{8-bit number}}]
      \gr{internal dimen}; the control sequence itself
      is a~\gr{box dimension}.
      Height of the box in a box register. 
Chapter~\ref{cschap:ht}.

\item [\cs{hyphenation\gr{general text}}]
      \gr{hyphenation assignment}
      Define hyphenation exceptions for the current value of \cs{language}.
Chapter~\ref{cschap:hyphenation}.

\item [\cs{hyphenchar\gr{font}}]
      \gr{internal integer}
      Number of the character  behind which a 
      \verb-\discretionary{}{}{}- is inserted.
Chapter~\ref{cschap:hyphenchar},\ref{cschap:hyphenchar2}.

\item [\cs{hyphenpenalty}]
      \gr{integer parameter}
      Penalty associated with break at a discretionary in the general case. 
      Default~\n{50} in plain \TeX.
Chapter~\ref{cschap:hyphenpenalty}.

\item [\cs{if\gr{token$_1$}\gr{token$_2$}}]
      \gr{expandable command}
      Test equality of character codes. 
Chapter~\ref{cschap:if}.

\item [\cs{ifcase\gr{number}\gr{case$_0$}\cs{or}\n{...}\cs{or}\gr{case$_n$}\cs{else}\gr{other cases}\cs{fi}}]
      \gr{expandable command}
      Enumerated case statement.
Chapter~\ref{cschap:ifcase}.

\item [\cs{ifcat\gr{token$_1$}\gr{token$_2$}}]
      \gr{expandable command}
      Test whether two characters have the same category code.
Chapter~\ref{cschap:ifcat}.

\item [\cs{ifdim\gr{dimen$_1$}\gr{relation}\gr{dimen$_2$}}]
      \gr{expandable command}
      Compare two dimensions. 
Chapter~\ref{cschap:ifdim}.

\item [\cs{ifeof\gr{4-bit number}}]
      \gr{expandable command}
      Test whether a file has been fully read, or does not exist.
Chapter~\ref{cschap:ifeof}.

\item [\cs{iffalse}]
      \gr{expandable command}
      This test is always false.
Chapter~\ref{cschap:iffalse}.

\item [\cs{ifhbox\gr{8-bit number}}]
      \gr{expandable command}
      Test whether a box register contains a horizontal box.
Chapter~\ref{cschap:ifhbox}.

\item [\cs{ifhmode}]
      \gr{expandable command}
      Test whether the current mode is (possibly restricted) horizontal mode.
Chapter~\ref{cschap:ifhmode}.

\item [\cs{ifinner}]
      \gr{expandable command}
      Test whether the current mode is an internal mode.
Chapter~\ref{cschap:ifinner}.

\item [\cs{ifmmode}]
      \gr{expandable command}
      Test whether the current mode is (possibly display) math mode.
Chapter~\ref{cschap:ifmmode}.

\item [\cs{ifnum\gr{number$_1$}\gr{relation}\gr{number$_2$}}]
      \gr{expandable command}
      Test relations between numbers.
Chapter~\ref{cschap:ifnum}.

\item [\cs{ifodd\gr{number}}]
      \gr{expandable command}
      Test whether a number is odd.
Chapter~\ref{cschap:ifodd}.

\item [\cs{iftrue}]
      \gr{expandable command}
      This test is always true.
Chapter~\ref{cschap:iftrue}.

\item [\cs{ifvbox\gr{8-bit number}}]
      \gr{expandable command}
      Test whether a box register contains a vertical box. 
Chapter~\ref{cschap:ifvbox}.

\item [\cs{ifvmode}]
      \gr{expandable command}
      Test whether the current mode is (possibly internal) vertical mode.
Chapter~\ref{cschap:ifvmode}.

\item [\cs{ifvoid\gr{8-bit number}}]
      \gr{expandable command}
      Test whether a box register is empty.
Chapter~\ref{cschap:ifvoid},\ref{cschap:ifvoid2}.

\item [\cs{ifx\gr{token$_1$}\gr{token$_2$}}]
      \gr{expandable command}
      Test equality of macro expansion, or equality of character code and
      category code.
Chapter~\ref{cschap:ifx}.

\item [\cs{ignorespaces}]  
      \gr{primitive command}
      Expands following tokens until something other
      than a~\gr{space token} is found.
Chapter~\ref{cschap:ignorespaces}.

\item [\cs{immediate}] 
      \gr{primitive command}
      Prefix to have output operations executed right away.
Chapter~\ref{cschap:immediate}.

\item [\cs{indent}]
      \gr{primitive command}
      Switch to horizontal mode and insert box with width \cs{parindent}.
      This command is automatically inserted before a
      \gr{horizontal 
      command} in vertical mode.
Chapter~\ref{cschap:indent}.

\item [\cs{input\gr{file name}}]
      \gr{expandable command}
      Read a specified file as \TeX\ input. 
Chapter~\ref{cschap:input}.

\item [\cs{inputlineno}]
      \gr{internal integer}
      (\TeX3 only) 
      Number of the current input line.
Chapter~\ref{cschap:inputlineno}.

\item [\cs{insert\gr{8-bit number}\lb\gr{vertical mode material}\rb}]
      \gr{primitive command}
      Start an insertion item.
Chapter~\ref{cschap:insert}.

\item [\cs{insertpenalties}]
      \gr{special integer}
      Total of penalties for split insertions.
      Inside the output routine the number of held-over insertions.
Chapter~\ref{cschap:insertpenalties}.

\item [\cs{interlinepenalty}]
      \gr{integer parameter}
      Penalty for breaking a page between lines of a paragraph. 
      Default~\n{0} in plain \TeX.
Chapter~\ref{cschap:interlinepenalty}.

\item [\cs{jobname}]
      \gr{expandable command}
      Name of the main \TeX\ file being processed.
Chapter~\ref{cschap:jobname}.

\item [\cs{kern\gr{dimen}}]
      \gr{kern}
      Add a kern item of the specified
      \gr{dimen} to the list;
      this can be used both in horizontal and vertical mode.
Chapter~\ref{cschap:kern}.

\item [\cs{language}]
      \gr{integer parameter} 
      (\TeX3 only)
      Choose a set of hyphenation patterns and exceptions.
Chapter~\ref{cschap:language}.

\item [\cs{lastbox}]
      \gr{box}
      Register containing the last element  added to the current list, 
      if this was a box.
Chapter~\ref{cschap:lastbox}.

\item [\cs{lastkern}]
      \gr{internal dimen}
      If the last item on the list was a kern, the size of this.
Chapter~\ref{cschap:lastkern}.

\item [\cs{lastpenalty}]
      \gr{internal integer}
      If the last item on the list was a penalty, the value of this.
Chapter~\ref{cschap:lastpenalty}.

\item [\cs{lastskip}]
      \gr{internal glue} or \gr{internal muglue}.
      If the last item on the list was a skip, the size of this.
Chapter~\ref{cschap:lastskip}.

\item [\cs{lccode\gr{8-bit number}}]
      \gr{internal integer}; the control sequence itself
      is a~\gr{codename}.
      Access the
      character code that is the lowercase variant of a given code.
Chapter~\ref{cschap:lccode}.

\item [\cs{leaders\gr{box or rule}\gr{vertical/horizontal/mathematical skip}}]
      \gr{leaders}
      Fill a specified amount of space with a rule or copies of box.
Chapter~\ref{cschap:leaders}.

\item [\cs{left}]
      \gr{primitive command}
      Use the following character as an open delimiter.
Chapter~\ref{cschap:left}.

\item [\cs{lefthyphenmin}]
      \gr{integer parameter} 
      (\TeX3 only)
      Minimum number of characters before a hyphenation.
Chapter~\ref{cschap:lefthyphenmin}.

\item [\cs{leftskip}]
      \gr{glue parameter}
      Glue that is placed to the left of all lines.
Chapter~\ref{cschap:leftskip}.

\item [\cs{leqno\gr{math mode material}\n{\char36\char36}}]
      \gr{eqno}
      Place a left equation number in a display formula.
Chapter~\ref{cschap:leqno}.

\item [\cs{let\gr{control sequence}\gr{equals}\gr{token}}]
      \gr{let assignment}
      Define a control sequence to a token, assign its meaning
      if the token is a command or macro. 
Chapter~\ref{cschap:let}.

\item [\cs{limits}]
      \gr{primitive command}
      Place limits over and under a large operator.
      This is the default position in display style.
Chapter~\ref{cschap:limits}.

\item [\cs{linepenalty}]
      \gr{integer parameter}
      Penalty value associated with each line break. 
      Default~\n{10} in plain \TeX.
Chapter~\ref{cschap:linepenalty}.

\item [\cs{lineskip}] 
      \gr{glue parameter}
      Glue added if distance between bottom and top of neighbouring boxes 
      is less than \cs{lineskiplimit}.
      Default~\n{1pt} in plain \TeX.
Chapter~\ref{cschap:lineskip}.

\item [\cs{lineskiplimit}]
      \gr{dimen parameter}
      Distance to be maintained between the bottom and top of 
      neighbouring boxes on a vertical list.
      Default~\n{0pt} in plain \TeX.
Chapter~\ref{cschap:lineskiplimit}.

\item [\cs{long}]
      \gr{prefix}
      Indicate that the arguments of the macro to be defined  
      are allowed to contain \cs{par} tokens.
Chapter~\ref{cschap:long}.

\item [\cs{looseness}] 
      \gr{integer parameter}
      Number of lines by which this paragraph has to be made longer 
      (or, if negative, shorter) than it would be ideally.
Chapter~\ref{cschap:looseness}.

\item [\cs{lower\gr{dimen}\gr{box}}]
      \gr{primitive command}
      Adjust vertical positioning of a box in horizontal mode. 
Chapter~\ref{cschap:lower}.

\item [\cs{lowercase\gr{general text}}]
      \gr{primitive command}
      Convert the argument to its lowercase form.
Chapter~\ref{cschap:lowercase}.

\item [\cs{mag}]
      \gr{integer parameter}
      1000 times the magnification of the document.
      Default \n{1000} in \IniTeX.
Chapter~\ref{cschap:mag}.

\item [\cs{mark\gr{general text}}]
      \gr{primitive command}
      Specify a mark text.
Chapter~\ref{cschap:mark}.

\item [\cs{mathaccent\gr{15-bit number}\gr{math field}}]
      \gr{primitive command}
      Place an accent in math mode.
Chapter~\ref{cschap:mathaccent},\ref{cschap:mathaccent2}.

\item [\cs{mathbin\gr{math field}}]
      \gr{math atom}
      Let the following \gr{math field} function 
      as a binary operation.
Chapter~\ref{cschap:mathbin}.

\item [\cs{mathchar\gr{15-bit number}}]
      \gr{primitive command}
      Explicit denotation of a mathematical character.
Chapter~\ref{cschap:mathchar}.

\item [\cs{mathchardef\gr{control sequence}\gr{equals}\gr{15-bit number}}]
      \gr{shorthand definition}
      Define a control sequence to be a synonym for
      a~math character code.
Chapter~\ref{cschap:mathchardef}.

\item [\cs{mathchoice\lb\SerifFont {\it D\/\rb\lb T\/\rb\lb S\/\rb\lb SS\/}\rb}]
      \gr{primitive command}
      Give four variants of a formula for the four styles
      of math typesetting.
     Chapter~\ref{cschap:mathchoice}.
 
\item [\cs{mathclose\gr{math field}}]
      \gr{math atom}
      Let the following \gr{math field} function
      as a closing symbol.
Chapter~\ref{cschap:mathclose}.

\item [\cs{mathcode\gr{8-bit number}}]
      \gr{internal integer}; the control sequence itself
      is a~\gr{codename}.
      Code of a character determining its treatment in math mode.
Chapter~\ref{cschap:mathcode}.

\item [\cs{mathinner\gr{math field}}]
      \gr{math atom}
      Let the following \gr{math field} function 
      as an inner formula.
Chapter~\ref{cschap:mathinner}.

\item [\cs{mathop\gr{math field}}]
      \gr{math atom}
      Let the following \gr{math field} function 
      as a large operator.
Chapter~\ref{cschap:mathop}.

\item [\cs{mathopen\gr{math field}}]
      \gr{math atom}
      Let the following \gr{math field} function 
      as an opening symbol.
Chapter~\ref{cschap:mathopen}.

\item [\cs{mathord\gr{math field}}]
      \gr{math atom}
      Let the following \gr{math field} function 
      as an ordinary object.
Chapter~\ref{cschap:mathord}.

\item [\cs{mathpunct\gr{math field}}]
      \gr{math atom}
      Let the following \gr{math field} function 
      as a punctuation symbol.
Chapter~\ref{cschap:mathpunct}.

\item [\cs{mathrel\gr{math field}}]
      \gr{math atom}
      Let the following \gr{math field} function as a relation.
Chapter~\ref{cschap:mathrel}.

\item [\cs{mathsurround}]
      \gr{dimen parameter}
      Kern amount placed before and after in-line formulas.
Chapter~\ref{cschap:mathsurround}.

\item [\cs{maxdeadcycles}]
      \gr{integer parameter}
      The maximum number of times that the output routine is allowed to
      be called without a \cs{shipout} occurring.
      \IniTeX\ default:~\n{25}.
Chapter~\ref{cschap:maxdeadcycles}.

\item [\cs{maxdepth}]
      \gr{dimen parameter}
      Maximum depth of the page box.
      Default~\n{4pt} in plain \TeX.
Chapter~\ref{cschap:maxdepth}.

\item [\cs{meaning}]
      \gr{expandable command}
      Give the meaning of a control sequence as a string of characters.
Chapter~\ref{cschap:meaning}.

\item [\cs{medmuskip}]
      \gr{muglue parameter}
      Medium amount of mu glue.
      Default value in plain \TeX: \n{4mu plus 2mu minus 4mu}
Chapter~\ref{cschap:medmuskip}.

\item [\cs{message\gr{general text}}]
      \gr{primitive command}
      Write a message to the terminal.
     Chapter~\ref{cschap:message}.
 
\item [\cs{mkern}]
      \gr{primitive command}
      Insert a kern measured in mu units.
Chapter~\ref{cschap:mkern}.

\item [\cs{month}]
      \gr{integer parameter}
      The month of the current job.
Chapter~\ref{cschap:month}.

\item [\cs{moveleft\gr{dimen}\gr{box}}]
      \gr{primitive command}
      Adjust horizontal positioning of a box in vertical mode. 
Chapter~\ref{cschap:moveleft}.

\item [\cs{moveright\gr{dimen}\gr{box}}]
      \gr{primitive command}
      Adjust horizontal positioning of a box in vertical mode. 
Chapter~\ref{cschap:moveright}.

\item [\cs{mskip}]
      \gr{mathematical skip}
      Insert glue measured in mu units.
Chapter~\ref{cschap:mskip}.

\item [\cs{multiply\gr{numeric variable}\gr{optional \n{by}}\gr{number}}]
      \gr{arithmetic assignment}
      Arithmetic command to multiply a
      \gr{numeric variable} (see \cs{advance}).
Chapter~\ref{cschap:multiply}.

\item [\cs{muskip\gr{8-bit number}}]
      \gr{internal muglue}; the control sequence itself
      is a~\gr{register prefix}.
      Access skips measured in mu units. 
Chapter~\ref{cschap:muskip}.

\item [\cs{muskipdef\gr{control sequence}\gr{equals}\gr{8-bit number}}]
      \gr{shorthand definition}; the control sequence
      itself is a~\gr{registerdef}.
      Define a control sequence to be a synonym for
      a~\cs{muskip} register.
Chapter~\ref{cschap:muskipdef}.

\item [\cs{newlinechar}]
      \gr{integer parameter}
      Number of the character that triggers a new line in
      \cs{write} and \cs{message} statements.
      Plain \TeX\ default~$-1$; \LaTeX\ default~$10$.
     Chapter~\ref{cschap:newlinechar}.
 
\item [\cs{noalign\gr{filler}\lb\gr{vertical (horizontal) mode material}\rb}]
    \gr{primitive command}
      Specify  vertical (horizontal)
      material   to be placed in between rows (columns) of
      an \cs{halign} (\cs{valign}).
Chapter~\ref{cschap:noalign}.

\item [\cs{noboundary}]
      \gr{horizontal command}
      (\TeX3 only)
      Omit implicit boundary character.
Chapter~\ref{cschap:noboundary}.

\item [\cs{noexpand\gr{token}}]
      \gr{expandable command}
      Do not expand the next token.
Chapter~\ref{cschap:noexpand}.

\item [\cs{noindent}] 
      \gr{primitive command}
      Switch to horizontal mode with an empty horizontal list.
Chapter~\ref{cschap:noindent}.

\item [\cs{nolimits}]
      \gr{primitive command}
      Place limits of a large operator as subscript and 
      superscript expressions.
      This is the default position in text style.
Chapter~\ref{cschap:nolimits}.

\item [\cs{nonscript}]
      \gr{primitive command}
      Cancel the next glue item if it occurs in 
      scriptstyle or scriptscriptstyle.
Chapter~\ref{cschap:nonscript}.

\item [\cs{nonstopmode}]
      \gr{interaction mode assignment}
      \TeX\ fixes errors as best it can,
      and performs an emergency stop
      when user interaction is needed.
Chapter~\ref{cschap:nonstopmode}.

\item [\cs{nulldelimiterspace}]
      \gr{dimen parameter}
      Width taken for empty delimiters. 
      Default~\n{1.2pt} in plain \TeX.
Chapter~\ref{cschap:nulldelimiterspace}.

\item [\cs{nullfont}]
      \gr{fontdef token}
      Name of an empty font that \TeX\ uses in emergencies.
Chapter~\ref{cschap:nullfont}.

\item [\cs{number\gr{number}}]
      \gr{expandable command}
      Convert a
      \gr{number} to decimal representation. 
Chapter~\ref{cschap:number}.

\item [\cs{omit}]
      \gr{primitive command}
      Omit the template for one alignment entry.
Chapter~\ref{cschap:omit}.

\item [\cs{openin\gr{4-bit number}\gr{equals}\gr{filename}}]
      \gr{primitive command}
      Open a stream for input.
Chapter~\ref{cschap:openin}.

\item [\cs{openout\gr{4-bit number}\gr{equals}\gr{filename}}]
      \gr{primitive command}
      Open a stream for output.
Chapter~\ref{cschap:openout}.

\item [\cs{or}]
      \gr{primitive command}
      Separator for entries of an \cs{ifcase}.
Chapter~\ref{cschap:or}.

\item [\cs{outer}]
      \gr{prefix}
      Indicate that the macro being defined 
      should occur on the outer level only.
Chapter~\ref{cschap:outer}.

\item [\cs{output}]
      \gr{token parameter}
      Token list with instructions for shipping out pages.
Chapter~\ref{cschap:output}.

\item [\cs{outputpenalty}]  
      \gr{integer parameter}
      Value of the penalty at the current page break,
      or $10\,000$ if the break was not at a penalty.
Chapter~\ref{cschap:outputpenalty},\ref{cschap:outputpenalty2}.

\item [\cs{over}]
      \gr{generalized fraction command}
      Fraction.
Chapter~\ref{cschap:over}.

\item [\cs{overfullrule}]
      \gr{dimen parameter}
      Width of the rule that is printed to indicate 
      overfull horizontal boxes.
      Plain \TeX\ default:~\n{5pt}.
Chapter~\ref{cschap:overfullrule}.

\item [\cs{overline\gr{math field}}]
      \gr{math atom}
      Overline the following \gr{math field}.
Chapter~\ref{cschap:overline}.

\item [\cs{overwithdelims\gr{delim$_1$}\gr{delim$_2$}}]
      \gr{generalized fraction command}
      Fraction with delimiters.
Chapter~\ref{cschap:overwithdelims}.

\item [\cs{pagedepth}]
      \gr{special dimen}
      Depth of the current page.
Chapter~\ref{cschap:pagedepth}.

\item [\cs{pagefilllstretch}]
      \gr{special dimen}
      Accumulated third-order stretch of the current page.
Chapter~\ref{cschap:pagefilllstretch}.

\item [\cs{pagefillstretch}]
      \gr{special dimen}
      Accumulated second-order stretch of the current page.
Chapter~\ref{cschap:pagefillstretch}.

\item [\cs{pagefilstretch}]
      \gr{special dimen}
      Accumulated first-order stretch of the current page.
Chapter~\ref{cschap:pagefilstretch}.

\item [\cs{pagegoal}]
      \gr{special dimen}
      Goal height of the page box. This starts at \cs{vsize},
      and is diminished by heights of insertion items.
Chapter~\ref{cschap:pagegoal}.

\item [\cs{pageshrink}]
      \gr{special dimen}
      Accumulated shrink of the current page.
Chapter~\ref{cschap:pageshrink}.

\item [\cs{pagestretch}]
      \gr{special dimen}
      Accumulated zeroth-order stretch of the current page.
Chapter~\ref{cschap:pagestretch}.

\item [\cs{pagetotal}]
      \gr{special dimen}
      Accumulated natural height of the current page.
Chapter~\ref{cschap:pagetotal}.

\item [\cs{par}]
      \gr{primitive command}
      Close off a paragraph and go into vertical mode.
Chapter~\ref{cschap:par}.

\item [\cs{parfillskip}]
      \gr{glue parameter}
      Glue that is placed between the last          
      element of the paragraph and the line end.
      Plain \TeX\ default:~\n{0pt plus 1fil}.
Chapter~\ref{cschap:parfillskip}.

\item [\cs{parindent}]
      \gr{dimen parameter}
      Size of the indentation box added in front of a paragraph.
Chapter~\ref{cschap:parindent},\ref{cschap:parindent2}.

\item [\cs{parshape}]
      \gr{internal integer}
      Command for general paragraph shapes: 
      \begin{disp}\cs{parshape}\gr{equals}$n$ $i_1$ $\ell_1$ $\ldots$
                $i_n$ $\ell_n$\end{disp}
      specifies a~number
      of lines~$n$, and $n$~pairs of an indentation and
      line length.
Chapter~\ref{cschap:parshape}.

\item [\cs{parskip}]
      \gr{glue parameter}
      Amount of glue added to vertical list when a paragraph starts; 
      default value \verb.0pt plus 1pt. in plain \TeX.
Chapter~\ref{cschap:parskip}.

\item [\cs{patterns\gr{general text}}]
      \gr{hyphenation assignment}
      Define a list of hyphenation patterns for the current
      value of \cs{language};  allowed only in \IniTeX.
Chapter~\ref{cschap:patterns}.

\item [\cs{pausing}]
      \gr{integer parameter}
      Specify that \TeX\ should pause after each line that is 
      read from a file.
Chapter~\ref{cschap:pausing}.

\item [\cs{penalty}]
      \gr{primitive command}
      Specify desirability of not breaking at this point.
Chapter~\ref{cschap:penalty},\ref{cschap:penalty2}.

\item [\cs{postdisplaypenalty}]
      \gr{integer parameter}
      Penalty placed in the vertical list below a display.
Chapter~\ref{cschap:postdisplaypenalty}.

\item [\cs{predisplaypenalty}]
      \gr{integer parameter}
      Penalty placed in the vertical list above a display.
      Plain \TeX\ default:~\n{10$\,$000}.
Chapter~\ref{cschap:predisplaypenalty}.

\item [\cs{predisplaysize}]
      \gr{dimen parameter}
      Effective width of the line preceding the display.
Chapter~\ref{cschap:predisplaysize}.

\item [\cs{pretolerance}]
      \gr{integer parameter}
      Tolerance value for a paragraph that uses no hyphenation. 
      Default~\n{100} in plain \TeX.
Chapter~\ref{cschap:pretolerance}.

\item [\cs{prevdepth}] 
      \gr{special dimen}
      Depth of the last box added to a vertical list as it is 
      perceived by \TeX.
Chapter~\ref{cschap:prevdepth}.

\item [\cs{prevgraf}] 
      \gr{special integer}
      The number of lines in the paragraph last
      added to the vertical list.
Chapter~\ref{cschap:prevgraf}.

\item [\cs{radical\gr{24-bit number}}]
      \gr{primitive command}
      Command for setting things such as root signs.
Chapter~\ref{cschap:radical}.

\item [\cs{raise\gr{dimen}\gr{box}}]
      \gr{primitive command}
      Adjust vertical positioning of a box in horizontal mode. 
Chapter~\ref{cschap:raise}.

\item [\cs{read\gr{number}\n{to}\gr{control sequence}}]
      \gr{simple assignment}
      Read a line from a stream into a control sequence.
Chapter~\ref{cschap:read}.

\item [\cs{relax}]
      \gr{primitive command}
      Do nothing.
Chapter~\ref{cschap:relax}.

\item [\cs{relpenalty}]
      \gr{integer parameter}
      Penalty for breaking after a binary relation, not enclosed
      in a subformula.
      Plain \TeX\ default:~\n{500}.
Chapter~\ref{cschap:relpenalty}.

\item [\cs{right}]
      \gr{primitive command}
      Use the following character as a closing delimiter.
Chapter~\ref{cschap:right}.

\item [\cs{righthyphenmin}]
      \gr{integer parameter} 
      (\TeX3 only) 
      Minimum number of characters after a hyphenation.
Chapter~\ref{cschap:righthyphenmin}.

\item [\cs{rightskip}]
      \gr{glue parameter}
      Glue that is placed to the right of all lines.
Chapter~\ref{cschap:rightskip}.

\item [\cs{romannumeral\gr{number}}]
      \gr{expandable command}
      Convert a positive
      \gr{number} to lowercase roman representation.
Chapter~\ref{cschap:romannumeral}.

\item [\cs{scriptfont\gr{4-bit number}}]
      \gr{family member}; the control sequence itself
      is a~\gr{font range}.
      Access the scriptfont of a family.
Chapter~\ref{cschap:scriptfont}.

\item [\cs{scriptscriptfont\gr{4-bit number}}]
      \gr{family member}; the control sequence itself
      is a~\gr{font range}.
      Access the scriptscriptfont of a family.
Chapter~\ref{cschap:scriptscriptfont}.

\item [\cs{scriptscriptstyle}]
      \gr{primitive command}
      Select the scriptscript style of math typesetting.
Chapter~\ref{cschap:scriptscriptstyle}.

\item [\cs{scriptspace}]
      \gr{dimen parameter}
      Extra space after subscripts and superscripts.
      Default~\n{.5pt}      in plain \TeX.
Chapter~\ref{cschap:scriptspace}.

\item [\cs{scriptstyle}]
      \gr{primitive command}
      Select the script style of math typesetting.
Chapter~\ref{cschap:scriptstyle}.

\item [\cs{scrollmode}]
      \gr{interaction mode assignment}  
      \TeX\ patches errors itself, but will ask the user for missing files.
Chapter~\ref{cschap:scrollmode}.

\item [\cs{setbox\gr{8-bit number}\gr{equals}\gr{box}}]
      \gr{simple assignment}
      Assign a box to a box register.
Chapter~\ref{cschap:setbox}.

\item [\cs{setlanguage\gr{number}}]
      \gr{primitive command} 
      (\TeX3 only) 
      Insert a whatsit resetting the current language
      to the \gr{number} specified.
Chapter~\ref{cschap:setlanguage}.

\item [\cs{sfcode\gr{8-bit number}}]
      \gr{internal integer}; the control sequence itself
      is a~\gr{codename}.
      Access the value of the \cs{spacefactor}
      associated with a character.
Chapter~\ref{cschap:sfcode}.

\item [\cs{shipout\gr{box}}]
      \gr{primitive command}
      Ship a box to the \n{dvi} file.
Chapter~\ref{cschap:shipout}.

\item [\cs{show\gr{token}}]
      \gr{primitive command}
      Display the meaning of a token on the screen.
Chapter~\ref{cschap:show}.

\item [\cs{showbox\gr{8-bit number}}]
      \gr{primitive command}
      Write the contents of a box to the log file.
Chapter~\ref{cschap:showbox}.

\item [\cs{showboxbreadth}]
      \gr{integer parameter}
      Number of successive elements that are shown when 
      \verb-\tracingoutput- is positive, each time a level is visited.
      Plain \TeX\ default:~\n{5}.
Chapter~\ref{cschap:showboxbreadth}.

\item [\cs{showboxdepth}] 
      \gr{integer parameter}
      The number of levels that are shown when 
      \verb-\tracingoutput- is positive.
      Plain \TeX\ default:~\n{3}.
Chapter~\ref{cschap:showboxdepth}.

\item [\cs{showlists}]
      \gr{primitive command}
      Write to the log file the contents of the partial lists 
      currently built in all modes.
Chapter~\ref{cschap:showlists}.

\item [\cs{showthe\gr{internal quantity}}]
      \gr{primitive command}
      Display on the terminal the result 
      of prefixing a token with \cs{the}.
Chapter~\ref{cschap:showthe}.

\item [\cs{skewchar\gr{font}}]
      \gr{internal integer}
      Font position of an after-placed accent.
     Chapter~\ref{cschap:skewchar}.
 
\item [\cs{skip\gr{8-bit number}}]
      \gr{internal glue}; the control sequence itself
      is a~\gr{register prefix}.
      Access skip registers
Chapter~\ref{cschap:skip}.

\item [\cs{skipdef\gr{control sequence}\gr{equals}\gr{8-bit number}}]
      \gr{shorthand definition}; the control sequence
      itself is a~\gr{registerdef}.
      Define a control sequence to be a synonym for
      a~\cs{skip} register.
Chapter~\ref{cschap:skipdef}.

\item [\cs{spacefactor}]
      \gr{special integer}
      1000 times the ratio by which the stretch component of the
      interword glue should be multiplied.
Chapter~\ref{cschap:spacefactor}.

\item [\cs{spaceskip}]
      \gr{glue parameter}
      Interword glue if non-zero.
Chapter~\ref{cschap:spaceskip}.

\item [\cs{span}]
      \gr{primitive command}
      Join two adjacent alignment entries, or (in preamble)
      expand the next token.
Chapter~\ref{cschap:span}.

\item [\cs{special\gr{general text}}]
      \gr{primitive command}
      Write a token list  to the \n{dvi} file.
Chapter~\ref{cschap:special}.

\item [\cs{splitbotmark}]
      \gr{expandable command}
      The last mark on a split-off page.
Chapter~\ref{cschap:splitbotmark}.

\item [\cs{splitfirstmark}]
      \gr{expandable command}
      The first mark on a split-off page.
Chapter~\ref{cschap:splitfirstmark}.

\item [\cs{splitmaxdepth}]
      \gr{dimen parameter}
      Maximum depth of a box split off by a \cs{vsplit} operation. 
      Default~\n{4pt} in plain \TeX.
Chapter~\ref{cschap:splitmaxdepth},\ref{cschap:splitmaxdepth2}.

\item [\cs{splittopskip}]
      \gr{glue parameter}
      Minimum distance between the top of what remains after a
      \cs{vsplit} operation, and the first item in that box.
      Default~\n{10pt} in plain \TeX.
Chapter~\ref{cschap:splittopskip}.

\item [\cs{string\gr{token}}]
      \gr{expandable command}
      Convert a token to a string of one or more characters. 
Chapter~\ref{cschap:string}.

\item [\cs{tabskip}]
      \gr{glue parameter}
      Amount of glue in between columns (rows) of an \cs{halign}
 \alt
      (\cs{valign}).
Chapter~\ref{cschap:tabskip}.

\item [\cs{textfont\gr{4-bit number}}]
      \gr{family member}; the control sequence itself
      is a~\gr{font range}.
      Access the textfont of a family.
Chapter~\ref{cschap:textfont}.

\item [\cs{textstyle}]
      \gr{primitive command}
      Select the text style of math typesetting.
Chapter~\ref{cschap:textstyle}.

\item [\cs{the\gr{internal quantity}}]
      \gr{primitive command}
      Expand the value of various quantities in \TeX\ into a string
      of (character) tokens.
Chapter~\ref{cschap:the}.

\item [\cs{thickmuskip}]
      \gr{muglue parameter}
      Large amount of mu glue. 
      Default value in plain \TeX: \n{5mu plus 5mu}.
Chapter~\ref{cschap:thickmuskip}.

\item [\cs{thinmuskip}]
      \gr{muglue parameter}
      Small amount of mu glue.
      Default value in plain \TeX: \n{3mu}.
Chapter~\ref{cschap:thinmuskip}.

\item [\cs{time}]
      \gr{integer parameter}
      Number of minutes after midnight that the current job started.
Chapter~\ref{cschap:time}.

\item [\cs{toks\gr{8-bit number}}]
      \gr{register prefix}
      Access a token list register.
Chapter~\ref{cschap:toks}.

\item [\cs{toksdef\gr{control sequence}\gr{equals}\gr{8-bit number}}]
      \gr{shorthand definition}; the control sequence
      itself is a~\gr{registerdef}.
      Assign a control sequence to
      a~\cs{toks} register.
Chapter~\ref{cschap:toksdef}.

\item [\cs{tolerance}]
      \gr{integer parameter}  
      Tolerance value for lines in a paragraph that does use hyphenation. 
      Default~\n{200} in plain \TeX, \n{10$\,$000} in \IniTeX.
Chapter~\ref{cschap:tolerance}.

\item [\cs{topmark}]
      \gr{expandable command}
      The last mark of the previous page.
Chapter~\ref{cschap:topmark}.

\item [\cs{topskip}]
      \gr{glue parameter}
      Minimum distance between the top of the page box
      and the baseline of the first box on the page. 
      Default~\n{10pt} in plain \TeX.
Chapter~\ref{cschap:topskip}.

\item [\cs{tracingcommands}]
      \gr{integer parameter}
      When this is~1, \TeX\ displays primitive commands executed; 
      when this is 2~or more the outcome of conditionals is also recorded.
Chapter~\ref{cschap:tracingcommands}.

\item [\cs{tracinglostchars}]
      \gr{integer parameter}      
      If this parameter is positive, \TeX\ gives      
      diagnostic messages whenever a character is accessed that      
      is not present in a font. Plain \TeX\ default:~1.
Chapter~\ref{cschap:tracinglostchars}.

\item [\cs{tracingmacros}]
      \gr{integer parameter}
      If this is~1, the log file shows expansion of macros 
      that are performed and the actual values of the arguments; 
      if this is 2~or more
      \gr{token parameter}s such as 
      \cs{output} and \cs{everypar} are also traced.
Chapter~\ref{cschap:tracingmacros}.

\item [\cs{tracingonline}]
      \gr{integer parameter}      
      If this parameter is positive, \TeX\ will write trace      
      information also to the terminal.
Chapter~\ref{cschap:tracingonline}.

\item [\cs{tracingoutput}]
      \gr{integer parameter}
      If this parameter is positive, the log file shows a dump of boxes 
      that are shipped to the \n{dvi} file.
Chapter~\ref{cschap:tracingoutput}.

\item [\cs{tracingpages}]
      \gr{integer parameter}      
      If this parameter is positive, \TeX\ generates      
      a trace of the page-breaking algorithm.
Chapter~\ref{cschap:tracingpages}.

\item [\cs{tracingparagraphs}]
      \gr{integer parameter}      
      If this parameter is positive, \TeX\ generates      
      a trace of the line-breaking algorithm.
Chapter~\ref{cschap:tracingparagraphs}.

\item [\cs{tracingrestores}]
      \gr{integer parameter}      
      If this parameter is positive, \TeX\ will report      
      all values that are restored when a group level ends.
Chapter~\ref{cschap:tracingrestores}.

\item [\cs{tracingstats}]
      \gr{integer parameter}      
      If this parameter is positive, \TeX\ reports at the      
      end of the job the usage of various internal arrays.
Chapter~\ref{cschap:tracingstats}.

\item [\cs{uccode\gr{8-bit number}}]
      \gr{internal integer}; the control sequence itself
      is a~\gr{codename}.
      Access
      the character code that is the uppercase variant of a given code.
Chapter~\ref{cschap:uccode}.

\item [\cs{uchyph}]
      \gr{integer parameter}
      Positive if hyphenating words starting with a capital 
      letter is allowed. 
      Plain \TeX\ default~1.
Chapter~\ref{cschap:uchyph}.

\item [\cs{underline\gr{math field}}]
      \gr{math atom}
      Underline the following \gr{math field}.
Chapter~\ref{cschap:underline}.

\item [\cs{unhbox\gr{8-bit number}}]
      \gr{horizontal command}
 \alt
      Unpack a box register containing a horizontal box,  
      appending the contents to the list, and emptying the register. 
Chapter~\ref{cschap:unhbox}.

\item [\cs{unhcopy\gr{8-bit number}}]
      \gr{horizontal command}
 \alt
      The same as \cs{unhbox}, but do not empty the register. 
Chapter~\ref{cschap:unhcopy}.

\item [\cs{unkern}]
      \gr{primitive command}
      Remove the last item of the list if this was a kern.
Chapter~\ref{cschap:unkern}.

\item [\cs{unpenalty}]
      \gr{primitive command}
      Remove the last item of the list if this was a penalty.
Chapter~\ref{cschap:unpenalty}.

\item [\cs{unskip}]
      \gr{primitive command}
      Remove the last item of the list if this was a skip.
Chapter~\ref{cschap:unskip}.

\item [\cs{unvbox\gr{8-bit number}}]
      \gr{vertical command}
 \alt
      Unpack a box register containing a vertical box, 
      appending the contents to the list, and emptying the register. 
Chapter~\ref{cschap:unvbox}.

\item [\cs{unvcopy\gr{8-bit number}}]
      \gr{vertical command}
 \alt
      The same as \cs{unvbox}, but do not empty the register. 
Chapter~\ref{cschap:unvcopy}.

\item [\cs{uppercase\gr{general text}}]
      \gr{primitive command}
      Convert the argument to its uppercase form.
Chapter~\ref{cschap:uppercase}.

\item [\cs{vadjust\gr{filler}\lb\gr{vertical mode material}\rb}]
      \gr{primitive command}
      Specify in horizontal mode material for the enclosing vertical list.
Chapter~\ref{cschap:vadjust}.

\item [\cs{valign\gr{box specification}\lb\gr{alignment material}\rb}]
      \gr{horizontal command}   
      Vertical alignment.   
Chapter~\ref{cschap:valign}.

\item [\cs{vbadness}]
      \gr{integer parameter}
      Threshold below which overfull and underfull vertical boxes 
      are not shown.
      Plain \TeX\ default:~\n{1000}.
Chapter~\ref{cschap:vbadness}.

\item [\cs{vbox\gr{box specification}\lb\gr{vertical material}\rb}]
      \gr{primitive command}
      Construct a vertical box with reference point on the last item. 
Chapter~\ref{cschap:vbox}.

\item [\cs{vcenter\gr{box specification}\lb\gr{vertical material}\rb}]
      \gr{primitive command}
      Construct a  vertical box vertically centred on the math axis.
Chapter~\ref{cschap:vcenter}.

\item [\cs{vfil}]
      \gr{vertical command}
      Vertical skip equivalent to \verb-\vskip 0cm plus 1fil-.
Chapter~\ref{cschap:vfil}.

\item [\cs{vfill}]
      \gr{vertical command}
      Vertical skip equivalent to \verb-\vskip 0cm plus 1fill-.
Chapter~\ref{cschap:vfill}.

\item [\cs{vfilneg}]
      \gr{vertical command}
      Vertical skip equivalent to \verb-\vskip 0cm minus 1fil-.
Chapter~\ref{cschap:vfilneg}.

\item [\cs{vfuzz}]
      \gr{dimen parameter} 
      Excess size that \TeX\ tolerates before it considers  
      a vertical box overfull.
      Plain \TeX\ default:~\n{0.1pt}.
Chapter~\ref{cschap:vfuzz}.

\item [\cs{voffset}]
      \gr{dimen parameter}
      Distance by which the page is shifted down from the reference point,
      which is one inch from the top of the page.
Chapter~\ref{cschap:voffset}.

\item [\cs{vrule}]
      \gr{horizontal command}
      Rule that spreads in vertical direction.
Chapter~\ref{cschap:vrule}.

\item [\cs{vsize}]
      \gr{dimen parameter}
      Height of the page box.
Chapter~\ref{cschap:vsize},\ref{cschap:vsize2}.

\item [\cs{vskip\gr{glue}}]
      \gr{vertical command}
      Insert in vertical mode a glue item.
Chapter~\ref{cschap:vskip}.

\item [\cs{vsplit\gr{8-bit number}\n{to}\gr{dimen}}]
      \gr{primitive command}
      Split off the top part of a vertical box. 
Chapter~\ref{cschap:vsplit},\ref{cschap:vsplit2}.

\item [\cs{vss}]
      \gr{vertical command}
      Vertical skip equivalent to \verb-\vskip 0cm plus 1fil minus 1fil-.
Chapter~\ref{cschap:vss}.

\item [\cs{vtop\gr{box specification}\lb\gr{vertical material}\rb}]
      \gr{primitive command}
      Construct a vertical box with reference point on the first item. 
Chapter~\ref{cschap:vtop}.

\item [\cs{wd\gr{8-bit number}}]
      \gr{internal dimen}; the control sequence itself
      is a~\gr{box dimension}.
      Width of the box in a box register.
Chapter~\ref{cschap:wd}.

\item [\cs{widowpenalty}]
      \gr{integer parameter}
      Additional penalty for breaking a page before 
      the last line of a paragraph. 
      Default~\n{150} in plain \TeX.
Chapter~\ref{cschap:widowpenalty}.

\item [\cs{write\gr{number}\gr{general text}}]
      \gr{primitive command}
      Generate a whatsit item containing
      a~token list to be written to the terminal or to a file. 
Chapter~\ref{cschap:write}.

\item [\cs{xdef}]
      \gr{def}
      Synonym for \verb-\global\edef-.
Chapter~\ref{cschap:xdef}.

\item [\cs{xleaders}]
      \gr{leaders}
      As \cs{leaders}, but with box leaders any excess space is 
      spread equally between the boxes.
Chapter~\ref{cschap:xleaders}.

\item [\cs{xspaceskip}]
      \gr{glue parameter}
      Interword glue if non-zero and \cs{spacefactor}${}\geq2000$.
Chapter~\ref{cschap:xspaceskip}.

\item [\cs{year}]
      \gr{integer parameter}
      The year of the current job.
Chapter~\ref{cschap:year}.

\end{glossinventory}

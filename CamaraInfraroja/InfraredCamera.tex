%\documentclass[options]{class}
\documentclass[12pt, twoside]{report}

%Paquete de Idioma
\usepackage[spanish]{babel}

%Codificación Alfabeto
\usepackage[utf8]{inputenc}

%Codificación de Fuente
\usepackage[T1]{fontenc}

%Índice
\usepackage{makeidx}

%Gráficos
\usepackage{graphicx}
\usepackage{float} 
%\usepackage{xcolor} 

%Matemática
\usepackage{amsmath}
\usepackage{amsfonts}
\usepackage{amssymb}
\usepackage{amstext} 

%Estilo de Página Numeración superior
%\pagestyle{headings}

%un estilo propio
\usepackage{fancyhdr}
\setlength{\headheight}{15pt}

\pagestyle{fancy}
\renewcommand{\chaptermark}[1]{ \markboth{\chaptername\ \thechapter: #1}{} }
\renewcommand{\sectionmark}[1]{ \markright{ Sección \thesection. #1}{} }

\fancyhf{}
\fancyhead[LE,RO]{\thepage}
\fancyhead[RE]{\textit{ \nouppercase{\leftmark}} }
\fancyhead[LO]{\textit{ \nouppercase{\rightmark}} }
\fancyfoot[CE]{\textit{\textcopyright 2014 Laboratorio de Sistemas Embebidos\\
	                    UPAEP} }
\fancyfoot[CO]{\textit{LSE002-2014 \\
		Elaboró: Dr. Casimiro Gómez González y Dr. Aurelio Horacio Heredia} }	            
\fancypagestyle{plain}{ %
	\fancyhf{} % remove everything
	\renewcommand{\headrulewidth}{0pt} % remove lines as well
	\renewcommand{\footrulewidth}{0pt}
}

%Hiperlinks \href{url}{text}
\usepackage[pdftex]{hyperref}

\usepackage{cite} % para contraer referencias

%Titulo
\title{LSE001-2014: Sistemas Operativos en Tiempo Real}
\author{Dr. Casimiro Gómez González y Dr. Aurelio Horacio Heredia\\
	Facultad de Electrónica, UPAEP\\
               correo: casimiro.gomez@upaep.mx\\
               Tel: 222 229 9428}
\date{Otoño 2014}

\begin{document}

\maketitle

\chapter*{Prólogo}

En el presente reporte se realiza un estudio de los sistemas operativos en tiempo real con el objetivo de aplicarlos en las tarjetas embebidas con microcontroladores ARM. Se ha elegido la plataforma \textit{CMSIS-RTOS RTX} de \textbf{Keil} y el microcontrolador \textit{LPC4357}. Se describe la aplicación de los sistemas operativos en tiempo real para un satélite \textbf{CubeSat}. Este documento se inicio en el otoño 2014 y estará en actualización continua.


\begin{flushright}
	
	Los autores\\
	Casimiro Gómez González\\
	Aurelio Horacio Heredia
\end{flushright}

\tableofcontents


\chapter{Introducción}

Los microprocesadores arribaron en 1970. Los sistemas operativos encontraron aplicación rápida en las sistemas basados en microprocesador. Para mediados de 1980 pocas de estas implementaciones usadas se pueden describir como sistemas operativos en tiempo real diseñados formalmente.Dos factores afectan la aceptación de los sistemas operativos en tiempo real, uno debido a los limites de la máquina, y los otros a la cultura de diseño alrededor de los microcontroladores. Los primeros microprocesador estaban bastante limitados en sus habilidades computacionales, velocidad de operación y capacidades de memoria. Tratar de establecer la estructura de un sistema operativo bajo estas bases es muy difícil. Sin embargo, la mayoría de estos sistemas operativos embebidos programados tienen poco o ningún relación 

\bibliographystyle{acm}
\bibliography{../bibliografia}
\end{document}
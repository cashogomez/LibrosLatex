\documentclass[12pt]{book}

\usepackage[utf8]{inputenc}
\usepackage[T1]{fontenc}
\usepackage{geometry}
\usepackage{graphicx}
\usepackage[spanish]{babel}
\usepackage{amsthm}
\usepackage{amsmath}
\usepackage{calrsfs}
\usepackage{trfsigns}

\newtheorem{thm}{Teorema}[section]
\theoremstyle{definition}
\newtheorem{dfn}{Definición}[section]
\theoremstyle{remark}
\newtheorem{note}{Nota}[section]
\theoremstyle{plain}
\newtheorem{lem}[thm]{Lema}

\geometry{letterpaper}



\title{Teoría de Fluidos}
\author{Dr. Casimiro Gómez González\\
	Facultad de Electrónica, UPAEP\\
               correo: casimiro.gomez@upaep.mx\\
               Tel: 222 229 9428}
\date{Primavera 2010}

\begin{document}
\frontmatter
\maketitle


\chapter{Prólogo}

El presente material es una recopilación de varios autores y de estudios propios realizados en la teoría de fluidos


\begin{flushright}

El autor\\
Casimiro Gómez González\\
Doctor en Ingeniería Mecatrónica
\end{flushright}

\tableofcontents

\mainmatter
\chapter{Cinemática}
La cinemática es la rama de la mecánica que trata con cantidades que involucran el espacio y el tiempo solamente. La cinematica estudia variables tales como el desplazamiento, la velocidad, la aceleración, la deformación y la rotación de elementos de fluidos sin referirse a la fuerza responsable de tal movimiento. 
La convención utilizada en este material es que el número de subíndices indica el orden del tensor. La dirección en el sistema de coordenadas cartesianas seran denotadas por $(x,y,z)$ y las componente de la velocidad se denotan por $(u,v,w)$. Cuando se utiliza la notación tensorial, las direcciones cartesianas se denotan por $(x_1, x_2, x_3)$, con las componentes de la  velocidades correspondientes $(u_1, u_2, u_3)$. Las coordenadas polares planares se denotan por $(r, \theta)$, con $u_r$ y $u_\theta$ como las componentes de velocidad respectivamente. Las coordenadas cilíndricas se denotan por $(R, \varphi, x)$ con $(u_R, u_\varphi , u_x)$ como las componentes de la velocidad correspondiente. Las coordenadas esféricas se denotan por $(r, \theta, \varphi)$, con $(u_r, u_\theta, u_\varphi)$ como las componentes de la velocidad respectivas.

\section{Especificación Lagrangiana y Euleriana}
Hay dos formas de describir el movimiento de un fluido. En la \textbf{descripción Lagrangiana}, uno esencialmente sigue la historia de partículas de fluidos  individuales. En consecuencia, las dos variables independientes son el tiempo y una etiqueta que se pone a las particulas de fluido. La etiqueta puede ser el vector de posición $\mathbf{x_0}$ de la partícula en algún instante de tiempo $t=0$. En esta descripción cualquier variable de flujo $F$ se expresa como $F(\mathbf{x_0}, t)$. En particular, la posición de la partícula es escrita como $\mathbf{x}(\mathbf{x_0}, t)$, la cual representa la localización de la partícula en $t$ cuya posición inicial fue $\mathbf{x_0}$ en $t=0$.

En la \textbf{descripción Euleriana}, uno se concentra en lo que pasa en un punto espacial $\mathbf{x}$, asi que las variables independientes son $\mathbf{x}$ y $t$. Esto es, las variables de flujo se escriben como $F(\mathbf{x},t)$.

\section{Derivada Material}
Sea $F$ cualquier variable de campo, tal como la temperatura, velocidad o estres. Empleando las coordenadas Eulerianas $(x,y,x,t)$, calculamos la velocidad de cambio de $F$ en cada punto siguiendo una partícula de identidad fija. La tarea es representar un concepto esencialmente Lagrangiano en un lenguaje Euleriano.

Para incrementos arbitrarios e independientes $dx$ y $dt$, el incremento en $F(\mathbf{x}, t)$ es
\begin{equation}
\label{equ100}
d F = \frac{\partial F}{\partial t} d t + \frac{\partial F}{d x_i} d x_i
\end{equation}

en donde se implica la suma sobre el índice repetido. Ahora suponiendo que el incremento no es arbitrario, y esta asociado con una particula determinada. El incremento $d\mathbf{x}$ y $d t$ esta relacionado a las componentes de la velocidad por las tres relaciones indicadas por
\begin{equation}
\label{equ101}
d x_i = u_i dt
\end{equation}

sustituyendo la ecuación \ref{equ101} en la ecuación \ref{equ100} 
\begin{equation}
\label{equ102}
\frac{d F}{dt} = \frac{\partial F}{\partial t}  + u_i \frac{\partial F}{d x_i} 
\end{equation}

el cambio total $D/Dt$ generalmente es llamado \textbf{derivada material} para enfatizar el hecho de que la derivada se toma siguiendo un elemento del fluido.

\backmatter
\end{document}
